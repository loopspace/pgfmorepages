\documentclass[a5paper,9pt]{extarticle}

\usepackage[T1]{fontenc}
\usepackage[utf8]{inputenc}

\usepackage[top=1cm, bottom=1cm, left=1cm, right=1cm]{geometry}
\usepackage[brazil]{babel}
\usepackage{cantos}

\usepackage{pgfmorepages}
%\usepackage{lipsum}

\pgfpagesuselayout{8 on 4, book format, reverse second, single sided}[a4paper,landscape]

\begin{document}
\pagestyle{empty}

\folheto{Folheto de cantos \texttt{\#}06070809}

\begin{cantos}

\begin{canto}
1. Bendito sejas tu, Senhor de nossos pais, és pródigo de graças, ó Senhor.

\textbf{Glória ao Senhor, criador para sempre! \rep{2}}

2. Bendito sejas tu, ó verbo de Deus Pai, a morte que sofreste nos deu vida.

3. Bendito sejas tu, Espírito de Deus, operas na Igreja a salvação.
\end{canto}

\begin{canto}
1. Senhor, eu estou aqui, venho te pedir, piedade de mim! \rep{2}

2. Cristo, eu estou aqui, venho te pedir, piedade de mim! \rep{2}

3. Senhor, estamos aqui, viemos te pedir, piedade de nós! \rep{2}

4. Cristo, estamos aqui, viemos te pedir, piedade de nós! \rep{2}
\end{canto}

\begin{canto}
\textbf{Glória, glória, glória, aleluia. Glória, glória, glória, aleluia. / Glória, glória, glória a Deus nos altos céus, paz na terra a todos nós.}

1. Deus e Pai, nós nos louvamos (glória a Deus), / adoramos, bendizemos (glória a Deus), / damos glória ao vosso nome (glória a Deus), / vossos dons agradecemos.

2. Senhor nosso, Jesus Cristo (hosana hey), / unigênito do Pai (hosana ha), / vós de Deus, Cordeiro Santo (hosana hey), / nossas culpas perdoai.

3. Vós que estais junto do Pai (aleluia), / como nosso intercessor (aleluia), / acolhei nossos pedidos (aleluia), / atendei nosso clamor.

4. Vós somente sois o Santo (glória a Deus), / o Altíssimo Senhor (hosana ha), / com o Espírito divino (aleluia), / de Deus Pai no esplendor.
\end{canto}

\begin{canto}
1. Ó Trindade imensa e una, vossa força tudo cria, / vossa mão que rege os tempos, antes deles existia.

2. Vós, feliz, num gozo pleno, totalmente vos bastais. / Pura, simples, generosa, terra e espaços abraçais.

3. Pai, da graça fonte viva, luz da glória de Deus Pai, / Santo Espírito da vida, que no amor os enlaçais.
\end{canto}

\begin{canto}
1. Deus eterno, a vós louvor, glória à vossa majestade. / Anjos e homens com fervor, vos adoram, Deus Trindade! / Cante a terra com amor, santo, santo é o Senhor. / Cante a terra com amor, santo, santo é o Senhor.

2. Pai eterno, a criação que tirastes vós do nada, / repousando em vossa mão, um acorde imenso brada: / quem me fez foi vosso amor, glória a vós, Pai criador! / Quem me fez foi vosso amor, glória a vós, Pai criador!

3. Filho eterno, nosso irmão, vossa morte deu-nos vida, / vosso sangue, salvação, toda a Igreja, agradecida, / louva, exalta a vós, Jesus, glória canta a vossa cruz! / Louva, exalta a vós, Jesus, glória canta a vossa cruz!
\end{canto}

\begin{canto}
1. Cantar a beleza da vida, presente do amor sem igual: / missão do teu povo escolhido, Senhor vem livrar-nos do mal.

\textbf{Vem dar-nos teu Filho, Senhor, sustento no pão e no vinho, / e a força do Espírito Santo unindo teu povo a caminho.}

2. Falar do teu Filho às nações, vivendo como Ele viveu: / missão do teu povo escolhido, Senhor, vem cuidar do que é teu.

3. Viver o perdão sem medida, servir sem jamais condenar: / missão do teu povo escolhido, Senhor, vem conosco ficar.
\end{canto}

\begin{canto}
1. Por melhor que seja alguém, chega o dia em que há de faltar. / Só o Deus vivo a palavra mantém e jamais Ele há de falhar.

\textbf{Quero cantar ao Senhor, sempre, enquanto eu viver, / hei de provar seu amor, seu valor e seu poder.}

2. Nosso Deus põe-se de lado dos famintos e injustiçados, / dos pobres e oprimidos, dos injustamente vencidos.
\end{canto}

\begin{canto}
1. Todos convidados cheguem ao banquete do Senhor. / Festa preparada, bem participada, venham partilhar do pão do amor.

\textbf{Cristo pão dos pobres, juntos nesta mesa, pois a eucaristia faz a Igreja. \rep{2}}

2. Vejam quanta fome, muitos lares sem ternura e pão. / Dor e violência, quanta resistência, vamos acolher a cada irmão.
\end{canto}

\begin{canto}
1. O batismo nos torna missionários, e o pecado amarra nossos pés.

\textbf{Senhor, Senhor, tende piedade de nós. \rep{2}}

2. Confirmados na fé, nos enviastes, e não fomos fiéis na vocação.

3. Recebendo o Pão da Eucaristia, resistimos a vida partilhar.
\end{canto}

\begin{canto}
1. Glória a Deus nos altos céus, paz na terra a seus amados. / A vós louvam, Rei celeste, os que foram libertados.

\textbf{Glória a Deus lá nos céus, e paz aos seus, amém!}

2. Deus e Pai, nós vos louvamos, adoramos, bendizemos, / damos glória ao vosso nome, vossos dons agradecemos.

3. Senhor nosso, Jesus Cristo, unigênito do Pai, / vós de Deus, Cordeiro Santo, nossas culpas perdoai!

4. Vós, que estais junto do Pai, como nosso intercessor, / acolhei nossos pedidos, atendei nosso clamor.

5. Vós somente sois o Santo, o Altíssimo, o Senhor, / com o Espírito Divino de Deus Pai no esplendor.
\end{canto}

\begin{canto}
\textbf{Aleluia, aleluia, como o Pai me amou, assim também eu vos amei. / Aleluia, aleluia, como estou no Pai, permanecei em mim.}

Vós todos que sofreis, aflitos, vinde a mim. / Repouso encontrarão os vossos corações. / Dou graças a meu Pai que revelou ao pobre, ao pequenino, seu grande amor.
\end{canto}

\begin{canto}
1. Tanta gente vai andando na procura de uma luz, / caminhando na esperança se aproxima de Jesus. / No deserto sente fome e o Senhor tem compaixão. / Comunica sua palavra, vai abrindo o coração.

\textbf{Dai-lhes vós mesmos de comer, que o milagre vai acontecer. \rep{2}}

2. Quando o pão é partilhado, passa a ter gosto de amor, / quando for acumulado, gera morte, traz a dor. / Quando o pouco que nós temos se transforma em oblação, / o milagre da partilha serve a mesa dos irmãos.

3. No altar da Eucaristia o Senhor vem ensinar / que o amor é verdadeiro quando a vida se doar. / Peregrinos, caminheiros, vamos juntos como irmãos, / na esperança repartindo a palavra e o mesmo pão.
\end{canto}

\begin{canto}
\textbf{Eu sou o pão que vem do céu, quem crer em mim irá viver.}

1. Nós reconhecemos o Senhor, partindo o pão. / Mistério de amor, a nossa refeição.

2. O Senhor Jesus no Sacramento nos deixou / memorial da cruz: morte e ressurreição.

3. Tão grande mistério adoramos, neste altar. / Que nossa fé sustente o nosso caminhar.
\end{canto}

\begin{canto}
1. Santa mãe, Maria, nesta travessia, cubra-nos teu manto cor de anil, / guarda nossa vida, mãe Aparecida, Santa Padroeira do Brasil.

\textbf{Ave Maria, Ave Maria. \rep{2}}

2. Mulher peregrina, força feminina, a mais importante que existiu, / com justiça queres que nossas mulheres sejam construtoras do Brasil.
\end{canto}

\begin{canto}
\textbf{O Senhor é minha luz, ele é minha salvação. / O que é que eu vou temer? Deus é minha proteção. / Ele guarda a minha vida, eu não vou ter medo, não. / Ele guarda a minha vida, eu não vou ter medo, não.}

1. Quando os maus vêm avançando, procurando me acuar, / desejando ver meu fim, só querendo me matar. / Inimigos opressores é que vão se liqüidar. / Inimigos opressores é que vão se liqüidar.

2. Se um exército se armar contra mim, não temerei. / Meu coração está firme, e firme ficarei. / Se estourar uma batalha, mesmo assim, confiarei. / Se estourar uma batalha, mesmo assim, confiarei.
\end{canto}

\begin{canto}
\textbf{Maria, ó mãe cheia de graça, Maria protege os filhos teus! / Maria, Maria, nós queremos contigo estar nos céus!}

Aqui servimos a Igreja do teu Filho, sob o teu Imaculado Coração. / Dá-nos a bênção, e nós faremos de nossa vida uma constante oblação.
\end{canto}

\begin{canto}
1. Não sei se descobriste a encantadora luz, no olhar da mãe feliz que embala o novo ser. / Nos braços leva alguém, em forma de outro eu, vivendo agora em dois, se sente renascer.

\textbf{A mãe será capaz de se esquecer, ou deixar de amar algum dos filhos que gerou? / E se existir acaso tal mulher, Deus se lembrará de nós em seu amor.}

2. O amor de mãe recorda o amor de nosso Deus, tomou se povo ao colo, quis nos atrair. / Até a ingratidão inflama seu amor, um Deus apaixonado busca a mim e a ti.
\end{canto}

\begin{canto}
1. Convite gentil não sei desprezar, que importa o que foi? Eu vim pra curar. / Quem nega o perdão em nome da lei, não quer ver o mundo irmão.

\textbf{Perdão, ó Senhor, misericórdia! Perdão, Senhor Deus da vida! \rep{2}}

2. Não posso aceitar o zelo fatal do fogo do céu em troca do mal. / Eu vim me propor, não vim pra forjar resposta com tal furor.

3. Tecer elogios por coisa qualquer revela, afinal, o quanto se quer. / Porém, escutai: mais vale cumprir o quanto ensinei do Pai.
\end{canto}

\begin{canto}
1. Muito alegre eu te pedi o que era meu. Partir, um sonho tão normal. / Dissipei meus bens, o coração também. No fim, meu mundo era irreal.

\textbf{Confiei no teu amor e voltei. Sim, aqui é meu lugar. / Eu gastei teus bens, ó Pai, e te dou este pranto em minhas mãos.}

2. Mil amigos conheci, disseram adeus. Caiu a solidão em mim. / Um patrão cruel levou-me a refletir: meu Pai não trata um servo assim.

3. Nem deixaste-me falar da ingratidão, morreu, no abraço, o mal que eu fiz. / Festa, roupa nova, o anel, sandália aos pés: voltei à vida, sou feliz.
\end{canto}

\begin{canto}
\textbf{Imaculada, Maria de Deus, coração pobre acolhendo Jesus. / Imaculada, Maria do povo, Mãe dos aflitos que estão junto à cruz.}

1. Um coração que era Sim para a vida, um coração que era Sim para o irmão. / Um coração que era Sim para Deus. Reino de Deus renovando este chão.

2. Olhos abertos pra sede do povo, passo bem firme que o medo desterra, / mãos estendidas que os tronos renegam. Reino de Deus que renova esta terra.

3. Faça-se, ó Pai, vossa plena vontade: que os nossos passos se tornem memória / do amor fiel que Maria gerou. Reino de Deus atuando na história.
\end{canto}

\begin{canto}
Se um dia, caíres no caminho, não digas nunca a teu pobre coração: / ``És mau e traidor, ingrato e desleal, nem olhes mais para o céu, não tens perdão!''

\textbf{Rancor destrói um coração que errou. / Melhor usar de mansidão e amor.}
\end{canto}

\begin{canto}
\textbf{Procuro abrigo nos corações, de porta em porta desejo entrar. / Se alguém me acolhe com gratidão, faremos juntos a refeição. / Se alguém me acolhe com gratidão, faremos juntos a refeição.}

1. Eu nasci pra caminhar assim, dia e noite, vou até o fim. / O meu rosto, o forte sol queimou, meu cabelo o orvalho já molhou. / Eu cumpro a ordem do meu coração.

2. Vou batendo até alguém abrir. Não descanso, o amor me faz seguir. / É feliz quem ouve a minha voz, e abre a porta, entro bem veloz. / Eu cumpro a ordem do meu coração.

3. Junto à mesa vou sentar depois e faremos refeição, nós dois. / Sentirá seu coração arder e esta chama tenho de acender. / Eu cumpro a ordem do meu coração.
\end{canto}

\begin{canto}
1. Antes que eu te formasses dentro do seio de tua mãe, antes que tu nascesses, te conhecia e te consagrei. / Para ser meu profeta entre as nações eu te escolhi. Irás onde enviar-te e o que te mando proclamarás.

\textbf{Tenho que gritar, tenho que arriscar, ai de mim se não o faço. Como escapar de ti, como calar, se tua voz arde em meu peito? / Tenho que andar, tenho que lutar, ai de mim se não o faço. Como escapar de ti, como calar, se tua voz me queima dentro?}

2. Não temas arriscar-te porque contigo eu estarei, não temas anunciar-me, em tua boca eu falarei. / Entrego-te meu povo, vai arrancar e derrubar. Para edificares, destruirás e plantarás.
\end{canto}

\begin{canto}
1. Senhor, que vieste salvar os corações arrependidos.

\textbf{Piedade, piedade, piedade de nós. \rep{2}}

2. Ó Cristo, que vieste chamar os pecadores humilhados.

3. Senhor, que intercedeis por nós junto a Deus Pai que nos perdoa.
\end{canto}

\begin{canto}
\textbf{Glória, glória, anjos no céu cantam todos seu amor. / E na terra, homens de paz: Deus merece o louvor!}

1. Deus e Pai, nós vos louvamos, adoramos, bendizemos, / damos glória ao vosso nome, vossos dons agradecemos.

2. Senhor nosso, Jesus Cristo, unigênito do Pai, / vós de Deus, Cordeiro Santo, nossas culpas perdoai!

3. Vós, que estais junto do Pai, como nosso intercessor, / acolhei nossos pedidos, atendei nosso clamor.

4. Vós somente sois o Santo, o Altíssimo, o Senhor, / com o Espírito Divino de Deus Pai no esplendor.
\end{canto}

\begin{canto}
1. Bendito seja Deus Pai, do universo criador, / pelo pão que nós recebemos, foi de graça e com amor.

\textbf{O homem que trabalha faz a terra produzir. O trabalho multiplica os dons que nós vamos repartir.}

2. Bendito seja Deus Pai, do universo criador, / pelo vinho que nós recebemos, foi de graça e com amor.

3. E nós participamos da construção do mundo novo / com Deus, que jamais despreza nossa imensa pequenez.
\end{canto}

\begin{canto}
\textbf{O corpo que era dele, eu comerei agora, o sangue que era dele, meu será! / A vida que era dele, eu viverei agora, o sonho que era dele, meu será!}

1. A farinha molhada na água é o pão. A farinha molhada na fé é Jesus. / Eis o sonho que o mundo não quis entender: quem não comer, não viverá!

2. Muita uva amassada no pé é o vinho. Muita uva amassada na fé é Jesus. / Eis o sonho que o mundo não quis entender: quem não beber, não viverá!
\end{canto}

\begin{canto}
Ó Mãe do Perpétuo Socorro, nós hoje, em feliz louvação, / cantamos tuas glórias, Maria, e o amor do menino, teu Filho. / A vida nas mãos, no olhar compaixão, só graça é teu coração.

\textbf{Mãe da misericórdia, mãe do puro amor, / dá-nos, no céu, contigo louvar Jesus, nosso Redentor.}
\end{canto}

\begin{canto}
\textbf{Aleluia, aleluia, aleluia, aleluia. \rep{2}}

Alguém do povo exclama: ``Como é grande, ó Senhor, quem te gerou e alimentou.'' / Jesus responde: ``Ó mulher, pra mim é feliz quem soube ouvir a voz de Deus e tudo guardou.''
\end{canto}

\begin{canto}
1. A mesa santa que preparamos, mãos que se elevam a ti, ó Senhor. / O pão e o vinho, frutos da terra, duro trabalho, carinho e amor.

\textbf{Ô, ô, ô, recebe, Senhor! Ô, ô, recebe, Senhor!}

2. Flores, espinhos, dor e alegria, pais, mães e filhos diante do altar. / A nossa oferta em nova festa, a nossa dor vem, Senhor, transformar.

3. A vida nova, nova família, que celebramos aqui tem lugar. / Tua bondade vem com fartura, é só saber reunir, partilhar.
\end{canto}

\begin{canto}
1. Tu, te abeiraste da praia, não buscaste nem sábios nem ricos, somente queres que eu te siga.

\textbf{Senhor, tu me olhaste nos olhos, a sorrir, pronunciaste meu nome. / Lá na praia, eu larguei o meu barco, junto a Ti buscarei outro mar.}

2. Tu sabes bem que em meu barco eu não tenho nem ouro nem espadas, somente redes e o meu trabalho.

3. Tu, minhas mãos solicitas, meu cansaço que a outros descanse, amor que almeja seguir amando.
\end{canto}

\begin{canto}
\textbf{Viva a Mãe de Deus e nossa, sem pecado concebida. / Viva a Virgem Imaculada, a Senhora Aparecida.}

1. Aqui estão vossos devotos, cheios de fé incendida, / de conforto e de esperança, ó Senhora Aparecida.

2. Virgem Santa, Virgem bela, mãe amável, mãe querida, / amparai-nos, protegei-nos, ó Senhora Aparecida.
\end{canto}

\begin{canto}
\textbf{No meio da tua casa, recebemos, ó Deus, a tua graça. / Sem fim, nossa louvação, pois a justiça está toda em tuas mãos.}

1. Alegrai-vos no Senhor, quem é bom, venha louvar, peguem logo o violão e o pandeiro pra tocar. / Para ele um canto novo, vamos, gente, improvisar.

2. Ele cumpre o que promete, podem nele confiar, Ele ama o que é direito, e Ele sabe bem julgar. / Sua palavra fez o céu, fez a terra e fez o mar.
\end{canto}

\begin{canto}
\textbf{Mãe admirável, ó mãe peregrina, a tua visita aquece e ilumina / pois trazes contigo teu filho Jesus, que é vida, caminho, verdade e luz.}

1. Por nossa Judéia, ó mãe, com carinho, tu vens apressada, estás a caminho. / E onde tu chegas a paz faz morada, as portas se abrindo em cada chegada.

2. De teu Santuário, tu vens peregrina. A graça trazendo que lá se origina. / Ao dar-nos abrigo, transformas pro bem, nosso apostolado, abençoas também.
\end{canto}

\begin{canto}
1. Pelos prados e campinas verdejantes eu vou, é o Senhor que me leva a descansar. / Junto às fontes de águas puras repousantes eu vou, minhas forças o Senhor vai animar.

\textbf{Tu és, Senhor, o meu pastor, por isso nada em minha vida faltará. \rep{2}}

2. Nos caminhos mais seguros, junto dele eu vou e pra sempre o seu nome eu honrarei. / Se eu encontro mil abismos, nos caminhos eu vou, segurança sempre tenho em suas mãos.

3. Ao banquete em sua casa muito alegre eu vou, um lugar em sua mesa me preparou. / Ele unge minha fronte e me faz ser feliz, e transborda minha taça em seu amor.
\end{canto}

\begin{canto}
1. Eu te exaltarei, meu Deus e Rei, por todas as gerações. / És o meu Senhor, Pai que me quer no amor.

\textbf{Entoai, ação de graças! E cantai um canto novo! / Aclamai a Deus Javé, aclamai com amor e fé!}
\end{canto}

\begin{canto}
\textbf{Acolhe os oprimidos em sua casa, ó Senhor, é seu abrigo! / Só Ele se faz temer, pois a seu povo dá força e poder!}

1. A nação que Ele governa é feliz com tal Senhor, / lá do céu Ele vê tudo, vê o homem e seu valor. / Fez o nosso coração forte e contemplador.

2. O que dá a vitória ao rei, não é ter muitos soldados. / O valente não se livra por sua força ou seus cuidados. / Quem confia nos cavalos vai, no fim, ser derrotado.
\end{canto}

\begin{canto}
1. Um dia escutei teu chamado, divino recado batendo no coração. / Deixei deste mundo as promessas e fui bem depressa no rumo da tua mão.

\textbf{Tu és a razão da jornada, tu és minha estrada, meu guia, meu fim. / No grito que vem do teu povo, te escuto de novo, chamando por mim.}

2. Os anos passaram ligeiro, me fiz um obreiro do Reino de paz e amor. / Nos mares do mundo navego, e às redes me entrego, tornei-me teu pescador.
\end{canto}

\begin{canto}
\textbf{Aleluia, aleluia, aleluia, aleluia, aleluia, aleluia. \rep{2}}

Rendei graças ao Senhor porque eterno é seu amor! \rep{2}
\end{canto}

\begin{canto}
\textbf{Tua Igreja é um corpo, cada membro é diferente. / E há no corpo, certamente, coração, ó meu Senhor. / Dele nasce a caridade, dom maior, mais importante. / Nele, enfim, achei radiante, minha vocação: o amor.}

1. Que loucura, não fizeste, vindo ao mundo nos salvar. / E depois que tu morreste, ficas vivo neste altar.

2. Os teus santos compreenderam teu amor sem dimensão. / E loucuras cometeram, em sua própria vocação.

3. Sou pequeno, igual criança, cheio de limitações. / Mas é grande a esperança, sinto muitas vocações.
\end{canto}

\begin{canto}
1. Pelos pecados, erros passados, por divisões na tua Igreja, ó Jesus. / \emph{Senhor, piedade. Senhor, piedade. / Senhor, piedade, piedade de nós. \rep{2}}

2. Quem não te aceita, quem te rejeita, pode não crer por ver cristãos que vivem mal. / \emph{Cristo, piedade. Cristo, piedade. / Cristo, piedade, piedade de nós. \rep{2}}

3. Hoje se a vida é tão ferida, deve-se a culpa à indiferença dos cristãos. / \emph{Senhor, piedade. Senhor, piedade. / Senhor, piedade, piedade de nós. \rep{2}}
\end{canto}

\begin{canto}
1. Glória a Deus Pai eu canto porque fez o céu, a terra, o mar e a mim também.

\textbf{Eu canto glória a Deus nas alturas / e para nós eu peço o amor, a paz, o bem!}

2. Glória a Jesus eu canto porque veio ao mundo, por Maria, nos salvar.

3. Glória ao Amor eu canto, porque vive em mim, me ensina a amar e a ser feliz.
\end{canto}


\begin{canto}
1. Eu confesso a Deus e a vós, irmãos, tantas vezes pequei, não fui fiel. / Pensamentos e palavras, atitudes, omissões, por minha culpa, tão grande culpa.

\textbf{Senhor, piedade! Cristo, piedade! Tem piedade, ó Senhor! \rep{2}}

2. Peço à Virgem Maria, nossa Mãe, e a vós, meus irmãos, rogueis por mim / a Deus Pai, que nos perdoa e nos sustenta em sua mão, por seu amor, tão grande amor.
\end{canto}

\begin{canto}
Como Maria, agora vou ouvir o que Jesus quer hoje me dizer. / Feliz é quem sabe escutar a Deus no coração, eu quero, ó meu Senhor, te amar.

\textbf{Aleluia, aleluia, o Evangelho vamos nós ouvir! / Aleluia, aleluia, a Jesus queremos aplaudir.}
\end{canto}

\begin{canto}
\textbf{Aleluia, aleluia! Aleluia, aleluia!}

No princípio era a palavra, e a palavra se encarnou. / E nós vimos sua glória, seu amor nos libertou.
\end{canto}

\begin{canto}
1. Como vai ser? Nossa festa não pode seguir. / Tarde demais pra buscar outro vinho e servir.

\textbf{Em meio a todo sobressalto, é Maria quem sabe lembrar: / ``Se o meu Filho está presente, nada pode faltar''! / ``Se o meu Filho está presente, nada pode faltar''!}

2. Mas que fazer? Se tem água, tem vinho também. / Basta um sinal, e em Caná quem provou: ``tudo bem!''
\end{canto}

\begin{canto}
1. Na festa da vida sem par, Caná põe a mesa, pois não! / Na mesa não pode faltar nem vinho, nem risos, nem pão! / Maria, que é mãe, ali vai, os noivos têm mãe em Caná. / Jesus quer saber a hora do Pai, Maria lhe diz: ``É já!''

\textbf{Maria, Maria, vem pôr, mãe querida, Jesus, pão da vida, na mesa do altar. / Maria, Maria, sem ti não há festa, ó vem, fica nesta, pra nada faltar!}

2. O vinho já está bem no fim, sem ele alegria não há, / não pode ficar triste assim a festa do amor em Caná. / De manso, Maria correu, e diz a Jesus o que quer. / E o vinho sobrou, a festa cresceu, Deus fez, só por ti, mulher!

3. Escutem o que Ele disser e façam o que ele mandar, / assim esta santa mulher ensina a palavra escutar. / Nas talhas, a água se faz um vinho que espanta os hebreus. / Assim sempre tem união, festa e paz, o povo que escuta a Deus.
\end{canto}

\begin{canto}
\textbf{Ouviste a palavra de Deus, guardaste em teu coração, / feliz porque creste, Maria, por ti nos vem a salvação. \rep{2}}

1. Nas palavras da lei e os profetas tua alma sedenta bebia. / A esperança do povo na vinda de Deus que os famintos sacia.

2. Quando o anjo por Deus foi mandado dizer-te da escolha tão alta, / sendo mãe, tu quiseste ser serva do Deus que os humildes exalta.

3. Quando o viste nascer rejeitado, perseguido até a morte cruel, / tua fé trouxe a Páscoa da vida, pois Deus para sempre é fiel.
\end{canto}

\begin{canto}
Irmão sol com irmã luz, trazendo o dia pela mão. / Irmão céu, de intenso azul, a invadir o coração: aleluia.

\textbf{Irmãos, minhas irmãs, vamos cantar nesta manhã / pois renasceu mais uma vez a criação das mãos de Deus. / Irmãos, minhas irmãs, vamos cantar: aleluia, aleluia, aleluia.}
\end{canto}

\begin{canto}
1. Companheira Maria, perfeita harmonia entre nós e o Pai. / Modelo dos consagrados, nosso ``sim'' ao chamado do Senhor confirmai.

\textbf{Ave Maria, cheia de graça, plena de graça e beleza, queres com certeza que a vida renasça. / Santa Maria, Mãe do Senhor, que se fez pão para todos, criou mundo novo só por amor.}

2. Intercessora Maria, perfeita harmonia entre nós e o Pai. / Justiça dos explorados, combate o pecado, torna os homens iguais.
\end{canto}

\begin{canto}
1. Maria concebida sem culpa original, trouxeste a luz da vida na noite de Natal. / Tu foste imaculada na tua Conceição, ó mãe predestinada da nova criação.

\textbf{Maria da Assunção, escuta a nossa voz, e pede proteção a cada um de nós. \rep{2}}

2. Maria, mãe querida, sinal de eterno amor, no ventre deste vida e corpo ao Salvador. / Ao céu foste elevada por anjos do Senhor, na glória coroada, coberta de esplendor.
\end{canto}

\begin{canto}
Vou te oferecer a vida e tudo que eu já sei viver: / tempo e trabalho, amor que eu espalho, coisas que me fazem crer. / Vou te oferecer o pranto, aquilo que é meu sofrer. / Paz que ainda não sei e tudo o que errei, são coisas que me fazem crer. / Pão e vinho são sinais de teu amor, nele eu vou saber viver. / Alegria e dor, eu vou te oferecer, são coisas que me fazem crer. / Alegria e dor, eu vou te oferecer, são coisas que me fazem crer.
\end{canto}

\begin{canto}
\textbf{Ó tu, que és o Senhor da vida, recebe em tuas mãos a minha vida.}

1. A tua oferta nos dá coragem de nos doarmos para sevir.

2. No dia-a-dia, em ti buscamos a força que nos sustenta.

3. A tua graça nos ilumina, fiéis seremos ao teu amor.
\end{canto}

\begin{canto}
1. Esta é a ceia do Pai, vinde todos, tomai o alimento eterno. / Hoje desejo saciar vossa fome de paz, acolhei-me no coração!

\textbf{Aonde iremos nós, aonde iremos nós? Tu tens palavras de vida e amor! Aonde iremos nós, aonde iremos nós? Tu és o verdadeiro Santo de Deus!}

2. Toda a verdade falei, feito pão eu deixei o meu corpo na mesa. / Hoje desejo estar outra vez entre vós, acolhei-me no coração!
\end{canto}

\begin{canto}
\textbf{Deus, nosso Pai protetor, dá-nos hoje um sinal de tua graça! / Por teu ungido, ó Senhor, estejamos pra sempre em tua casa!}

1. Ó Senhor, põe teu ouvido bem aqui, pra me escutar. / Infeliz eu sou e pobre, vem depressa me ajudar. / Teu amigo eu sou, tu sabes, só em ti vou confiar.

2. Compaixão de mim, Senhor, eu te chamo, noite e dia, / vem me dar força e coragem e aumentar minha alegria. / Eu te faço minha prece, pois minha alma em ti confia.
\end{canto}

\begin{canto}
\textbf{A Bíblia é a palavra de Deus semeada no meio do povo, / que cresceu, cresceu e nos transformou, ensinando-nos viver um mundo novo.}

1. Deus é bom, nos ensina a viver, nos revela o caminho a seguir. / Só no amor partilhando seus dons, sua presença iremos sentir.

2. Somos povo, o povo de Deus, e formamos o reino de irmãos, / e a palavra que é viva nos guia e alimenta a nossa união.
\end{canto}

\begin{canto}
1. Numa terra distante daqui, um povo buscava sua libertação. / Este povo era um povo de escravos já sem esperança no seu coração. / Deste povo surgiu um profeta, de sua vida ao Senhor fez oferta. / Ao ouvir a Palavra de Deus que é amor, o seu povo libertou. / Ao ouvir a Palavra de Deus que é amor, o seu povo libertou.

2. Mas aqui, neste chão, nossa terra, um povo sofrido eleva suas mãos. / Fala alto o Senhor por suas vozes que clamam justiça e libertação. / Este povo também tem profeta, de sua vida ao Senhor faz oferta. / Escutando a Palavra de Deus lhe chamar, quer seu povo libertar. / Escutando a Palavra de Deus lhe chamar, quer seu povo libertar.
\end{canto}

\begin{canto}
\textbf{Feliz o homem que ama o Senhor e segue seus mandamentos. / O seu coração é repleto de amor, Deus mesmo é seu alimento.}

1. Feliz o que anda na lei do Senhor e segue o caminho que Deus lhe indicou: / terá recompensa no reino do céu porque muito amou.

2. Feliz quem se alegra em servir o irmão segundo os preceitos que Deus lhe ensinou: / verá maravilhas de Deus, o Senhor, porque muito amou.

3. Feliz quem confia na força do bem, seguindo os caminhos da paz e o perdão: / será acolhido nos braços do Pai porque muito amou.
\end{canto}

\begin{canto}
Maria de Nazaré, Maria me cativou. Fez mais forte a minha fé e por filho me adotou. / Às vezes, eu paro e fico a pensar e sem perceber me vejo a rezar. / E o meu coração se põe a cantar pra Virgem de Nazaré. / Menina que Deus amou e escolheu pra Mãe de Jesus, o Filho de Deus. / Maria que o povo inteiro elegeu Senhora e Mãe do céu.

\textbf{Ave Maria \rep{3}, Mãe de Jesus.}
\end{canto}

\begin{canto}
1. Maria, pura e santa, aos olhos do Senhor. / Por Deus fostes escolhida pra seres mãe da vida, mãe do Salvador.

\textbf{Ao sermos chamados como tu, seguimos a luz dos passos teus. / E estamos aqui dizendo ``sim'' ao nosso Deus. \rep{2}}

2. Tu fostes peregrina da nossa redenção. / Por onde tu andavas, a luz de Deus levavas no teu coração.
\end{canto}

\begin{canto}
1. Glória, glória a Deus nas alturas, nas criaturas, na história! / Glória, glória ao Deus criador, vivo amor entre os povos, glória! / Glória, glória ao Deus criador, vivo amor entre os povos, glória! / \emph{No sol, nas estrelas, na terra e no mar, glória, glória, aleluia! \rep{2}}

2. Glória, glória ao Cristo bendito, ressuscitou, é vitória! / Glória, oxalá, Deus presente, na vida da gente, glória! / Glória, oxalá, Deus presente, na vida da gente, glória! / \emph{Nas comunidades, na rua, no lar, glória, glória, aleluia! \rep{2}}

3. Glória, glória ao Espírito Santo, graça, esperança e memória! / Glória, glória à luz que alumia, alegria dos pobres, glória! / Glória, glória à luz que alumia, alegria dos pobres, glória! / \emph{No canto, na dança, na festa, no altar, glória, glória, aleluia! \rep{2}}
\end{canto}

\begin{canto}
\textbf{Quero que faças em mim segundo a tua palavra, Senhor. / Quero dizer sempre ``sim'' ao teu projeto de amor. / Quero dizer sempre ``sim'' ao teu projeto de amor.}

O anjo de Deus anunciou a Maria, e ela aceitou o que o anjo dizia.
\end{canto}

\begin{canto}
1. Ó Mãe, por intermédio do teu nome, queremos nossos dons oferecer. / O povo não tem pão e passa fome e espera nossa oferta acontecer.

\textbf{Maria, medianeira divinal, se pedes teu Jesus atenderá. / Repete o teu apelo maternal, assim como nas Bodas de Caná.}

2. Ó Mãe, por teu materno sentimento, queremos nossos dons oferecer. / O povo não tem vinho, está sedento e espera nossa oferta acontecer.
\end{canto}

\begin{canto}
1. Tu foste a primeira criatura que o corpo de Cristo recebeu. / Tão cheia de graça e toda pura, tu deste morada ao próprio Deus.

\textbf{Senhora, mãe da vida e da alegria, ensina a nos abrirmos para o amor. / Por meio desta santa Eucaristia, queremos ser os templos do Senhor.}

2. Belém se fechou quando pediste um simples lugar pra teu Jesus. Choraste de dor, mas assumiste, num rancho de ovelhas deste a luz.

3. Depois, bem no alto do Calvário, recebes o Cristo aos pés da cruz. / E o teu coração foi o Sacrário de toda a Paixão do teu Jesus.
\end{canto}

\begin{canto}
Ó minha Senhora e também minha Mãe, eu me ofereço inteiramente todo a Vós / e em prova da minha devoção eu hoje vos dou meu coração.

Consagro a Vós meus olhos, meus ouvidos, minha boca. Tudo o que sou, desejo que a Vós pertença. / Incomparável Mãe, guardai-me, defendei-me como filho e propriedade vossa, amém.
\end{canto}

\begin{canto}
\textbf{Quanto a nós, devemos gloriar-nos na Cruz de Nosso Senhor Jesus Cristo. / Que é nossa salvação, nossa vida, nossa esperança de ressurreição, / e pelo qual fomos salvos e libertos.}

1. Esta é a noite da Ceia Pascal, a Ceia em que o nosso Cordeiro se imolou.

2. Esta é a noite da Ceia do amor, a Ceia em que Jesus por nós se entregou.

3. Esta é a Ceia da nova aliança, aliança confirmada no sangue do Senhor.
\end{canto}

\begin{canto}
\textbf{Salve, ó Cristo obediente, salve amor onipotente. / Que te entregou à cruz, e te recebeu na luz.}

1. O Cristo obedeceu até a morte, humilhou-se e obedeceu o bom Jesus. / Humilhou-se e obedeceu, sereno e forte, humilhou-se e obedeceu até a cruz.
\end{canto}

\begin{canto}
\textbf{Prova de amor maior não há que doar a vida pelo irmão. \rep{2}}

1. Eis que eu vos dou o meu novo mandamento: ``Amai-vos uns aos outros como Eu vos tenho amado''.

2. Vós sereis os meus amigos se seguirdes meu preceito: ``Amai-vos uns aos outros como Eu vos tenho amado''.

3. Permanecei em meu amor e segui meu mandamento: ``Amai-vos uns aos outros como Eu vos tenho amado''.
\end{canto}

\begin{canto}
\textbf{Glória, glória nas alturas, paz e amor na terra aos homens. / Dêem-vos glória, criaturas, dêem-vos graças e louvores.}

1. Nós vos louvamos, ó Criador, vos bendizemos por vosso amor.

2. Nós vos louvamos, Senhor Jesus, vos aclamamos por vossa cruz.

3. Espírito Santo consolador, vós que dais vida e sois Senhor.
\end{canto}

\begin{canto}
\textbf{Pai, a Igreja vos pede só isto: vosso Espírito aqui derramai! / Pra me ungir testemunha de Cristo, e eu poder vos chamar Deus, meu Pai!}

1. Quero a graça da Sabedoria, ter Ciência, não ouro e poder. / Pra sorrir como Cristo sorria, porque o Pai faz o lírio crescer.

2. Quero o dom desse Espírito forte, que me ensina sofrer a cantar. / Serei vida onde o ódio é só morte, serei luz onde a treva reinar.

3. Quero o dom do Conselho bendito, quero a luz que nos faz discernir. / Quem cair, se levante contrito, quem amar siga a estrada a sorrir.
\end{canto}

\begin{canto}
\textbf{Aleluia, é o nosso canto, Jesus Cristo vai falar. / E o Espírito que é Santo é quem vai nos explicar.}

Santo e santificador, iluminai nossa mente e o fogo do vosso amor / encha o coração da gente, encha o coração da gente.
\end{canto}

\begin{canto}
1. Espírito Criador, com o Pai fazeis fecundo o solo imenso do mundo pra nos dar trigo e flor. / Bendito sois noite e dia por tão grande doação. Fonte sem fim de alegria, são matérias pro nosso pão.

2. Espírito Criador, foi dom de vossa bondade encher-nos de habilidade pro trabalho, Senhor. / Com o Pai Vós sois bendito, porque dais à nossa mão, com poder que é quase infinito, continuar a criação.
\end{canto}

\begin{canto}
1. Ó Senhor, tu me ungiste na fronte com o óleo que cura a ferida / para eu ir a qualquer horizonte, suavizando essas dores da vida.

\textbf{Mas pra dar tua Paz noite e dia, e estar sempre a serviço do irmão, / eu preciso da tua energia, eu preciso, Jesus, deste Pão!}

2. Bem na fronte, Senhor Deus, me ungiste com o óleo da santa alegria, / e eu serei o consolo do triste, e quem chora farei que sorria.

3. Bem na fronte me ungiste, Senhor, com o óleo capaz de ser luz. / Doravante, como ungido onde eu for, eu irei irradiar a Jesus.
\end{canto}

\begin{canto}
1. Senhor, vem dar-nos sabedoria, que faz ter tudo como Deus quis, / e assim faremos da Eucaristia o grande meio de ser feliz.

\textbf{Dá-nos, Senhor, esses dons, essa luz, e nós veremos que Pão é Jesus.}

2. Dá-nos, Senhor, o entendimento, que tudo ajuda a compreender, / para nós vermos como é alimento o pão e o vinho que Deus quer ser.

3. Senhor, vem dar-nos divina ciência, que, como o eterno, faz ver sem véus. / Tu vês por fora, Deus vê a essência, pensas que é pão, mas é nosso Deus.

4. Dá-nos, Senhor, o teu conselho, que nos faz sábios para guiar. / Homem, mulher, jovem e velho nós guiaremos ao Santo Altar.
\end{canto}

\begin{canto}
\textbf{Eu canto a alegria, Senhor / de ser perdoado no amor. \rep{2}}

Senhor, tende piedade de nós. \rep{2} / Cristo, tende piedade de nós. \rep{2} / 
Senhor, tende piedade de nós. \rep{2}
\end{canto}

\begin{canto}
1. O povo te chama de Nossa Senhora por causa de Nosso Senhor. / O povo te chama de Mãe e Rainha porque Jesus Cristo é o Rei do céu. / E por não te ver como desejaria, te vê com os olhos da fé, / por isso Ele coroa a tua imagem, Maria, por seres a mãe de Jesus, por seres a mãe de Jesus de Nazaré.

\textbf{Como é bonita uma religião que se lembra da mãe de Jesus, mais bonito é saber quem tu és. / Não és deusa, não és mais que Deus, mas depois de Jesus, o Senhor, neste mundo ninguém foi maior.}
\end{canto}

\begin{canto}
Por Deus chamado, escolhido, livre, sou comprometido com a construção de seu Reino entre nós. / Seguirei o meu Senhor, vivendo o Evangelho do amor. / Pobre, sou enriquecido por um amor feito pleno, comungando sempre a vontade do Pai. / Deixando tudo por Cristo, a ele só me reservo, para, disponível, servir o irmão. / Minha resposta ao chamado é radical, sem medida, numa adesão renovada, atual.
\end{canto}

\begin{canto}
1. És, Maria, a Virgem que sabe ouvir e acolher com fé a santa palavra de Deus. / Dizes ``sim'' e logo te tornas Mãe, dás à luz depois o Cristo que vem nos remir.

\textbf{Virgem que sabe ouvir o que o Senhor te diz. Crendo, geraste quem te criou. Ó Maria, tu és feliz. \rep{2}}
\end{canto}

\begin{canto}
1. Bendito sejas, ó Rei da glória, ressuscitado, Senhor da Igreja. Aqui trazemos as nossas ofertas.

\textbf{Vê com bons olhos nossas humildes ofertas. / Tudo que temos, seja pra ti, ó Senhor!}

2. Vidas se encontram no altar de Deus. Gente se doa, dom que se imola. Aqui trazemos as nossas ofertas.
\end{canto}

\begin{canto}
1. Estava a mãe dolorosa, ao pé da cruz, lacrimosa, enquanto o Filho pendia, enquanto o Filho pendia.

\textbf{Mãe de Jesus, transpassada de dores, aos pés da cruz, / rogai por nós, rogai por nós, rogai por nós a Jesus. / Rogai por nós, rogai por nós, rogai por nós a Jesus.}
\end{canto}

\begin{canto}
1. Na longa estrada da vida, tua gente sofrida em busca do amor, / percorre diversos caminhos de cravos e espinhos, de luta e de dor. / Em ti nossa gente confia e em romaria vem te contemplar, / mãezinha, consolo dos crentes, ensina essa gente a Jesus adorar.

\textbf{Mãe do Perpétuo Socorro, venho a ti e recorro, vem, ó Mãe, me valer! / Mãe, nosso eterno auxílio, vem nos dar o teu filho, Mãe, vem nos socorrer!}
\end{canto}

\end{cantos}

\end{document}
